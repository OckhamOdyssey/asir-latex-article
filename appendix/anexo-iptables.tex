\chapter{Aislamiento de subredes mediante iptables}
\label{anexo:iptables}

A continuación se muestra la salida del comando \texttt{iptables-save}, que nos permite ver la configuración completa de las iptables del sistema de forma que pueda aplicarse de nuevo fácilmente.

\begin{lstlisting}[language=Bash]
# Generated by iptables-save v1.8.7 on Sun May 22 14:28:28 2022
*filter
:INPUT ACCEPT [0:0]
:FORWARD ACCEPT [0:0]
:OUTPUT ACCEPT [0:0]
-A FORWARD -s 10.218.0.0/16 -d 10.218.0.1/32 -i tun0 -j ACCEPT
-A FORWARD
# De sistemas a todas
-A FORWARD -s 10.218.1.0/24 -d 10.218.0.0/16 -i tun0 -j ACCEPT
# De gerencia a todas
-A FORWARD -s 10.218.8.0/24 -d 10.218.0.0/16 -i tun0 -j ACCEPT
# De backend a API
-A FORWARD -s 10.218.2.0/24 -d 10.218.0.3 -i tun0 -j ACCEPT
# De backend a API
-A FORWARD -s 10.218.2.0/24 -d 10.218.0.3 -i tun0 -j ACCEPT
# Entre subnets
-A FORWARD -s 10.218.0.0/16 -d 10.218.0.0/16 -i tun0 -j DROP
COMMIT
# Completed on Sun May 22 14:28:28 2022
# Generated by iptables-save v1.8.7 on Sun May 22 14:28:28 2022
*nat
:PREROUTING ACCEPT [268:18582]
:INPUT ACCEPT [65:4102]
:OUTPUT ACCEPT [346:24723]
:POSTROUTING ACCEPT [431:30351]
-A POSTROUTING -s 10.218.0.0/16 -o ens4 -m comment --comment openvpn-nat-rule -j MASQUERADE
COMMIT
# Completed on Sun May 22 14:28:28 2022
\end{lstlisting}
% Al final de la línea 15 he borrado  --reject-with icmp-port-unreachable para que quepa en el cuadro