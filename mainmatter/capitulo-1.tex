\chapter{Introducción}
\label{chapter:introduction}
%%An introduction... \cite{example-article}

En los últimos años, la forma de organizar las instituciones ha cambiado drásticamente a causa de la pandemia de la COVID-19. El teletrabajo y la forzada digitalización de la sociedad por los confinamientos ha causado un avance acelerado al que muchas empresas no estaban preparadas, creando una red corporativa débil y propensa a ataques informáticos que, hasta ahora, no se creía que podían ser objetivo. 

Las compañías que ya formaban parte del sector tecnológico también se han visto afectadas, y es que la gran mayoría no disponían de un entorno adaptado al trabajo a distancia. Es por ello que este proyecto se enfocará en la adaptación de una red empresarial a un sistema más flexible, permitiendo a los empleados trabajar y acceder a los recursos de la empresa desde cualquier punto, evitando sacrificar la seguridad informática.
\section{La empresa}
La empresa ficticia con la que se trabajará se llama [REDACTADO], dedicada a la programación de aplicaciones adaptadas para las empresas cliente. Como parte del sector tecnológico, la corporación cuenta con un total de 27 empleados, la gran mayoría del departamento de ingeniería. 

\begin{table}[h!]
\centering
\begin{tabularx}{300pt}{|r|X|c|}
\hline
Departamento   & Equipo         & Cantidad \\ \hline
Ingeniería     & Sistemas       & 2        \\
               & Backend        & 7        \\
               & Frontend       & 6        \\ \hline
Diseño         & UI/UX          & 2        \\
               & Diseño gráfico & 3        \\ \hline
Administración & Contabilidad   & 2        \\
               & Ventas         & 4        \\
Dirección      & Gerente        & 1        \\ \hline
Total          &                & 27       \\ \hline
\end{tabularx}
\label{tab:tabla_personal}
\caption{Tabla de número de empleados por departamento}
\end{table}

Así mismo, la empresa cuenta con un consejo general formado por el gerente de la empresa y un empleado responsable de cada departamento, cuatro personas en total.
\subsection{Relación con la FCT}
A pesar de ser ficticia, la estructura de [REDACTADO] está fuertemente inspirada en la empresa donde realizo las prácticas, reduciendo la cantidad de empleados y departamentos para adaptarlo a este proyecto. El nicho de la empresa se ha generalizado y la organización del \textquote{consejo general} mencionado en el apartado anterior está basado en una simplificación de la estructura organizativa de otra institución en la que poseo experiencia.

El nombre seleccionado para la empresa tampoco ha sido elegido al azar. Poseo el dominio de [REDACTADO] ([REDACTADO].es) para realizar pruebas de despliegue de servicios para uso exclusivamente personal. Aprovecharé la posesión de este dominio para usarlo para la empresa del proyecto y poder hacer todos los pasos lo más veraz posible.
\section{Necesidades}
Durante la pandemia de la COVID-19, la empresa sufrió un cambio forzado en la estructuración de la empresa. Los empleados no podían llevarse los ordenadores de torre a sus casas y tuvieron que hacer uso de los suyos personales, manejando información crítica sin las correctas medidas de seguridad. Ahora, con algunos empleados aún trabajando en remoto, la empresa nos ha solicitado adaptar su red informática a una más flexible y segura. Con un plazo máximo de tres meses, se requerirá que los empleados cuenten con dispositivos móviles adecuados a su puesto de trabajo, controlados y auditados por el departamento de sistemas, y con medidas de seguridad suficientes.

Hasta ahora, muchos de los datos, como la gestión del inventario, se mantenían en hojas de cálculo almacenadas en carpetas compartidas por un servidor local. Para paliarlo, la empresa solicita un servidor con la información accesible también por los ordenadores que se encuentren físicamente fuera de la oficina.

Para todo ello, el equipo directivo pone un presupuesto máximo de 500 € mensuales para gastos relacionados con este proyecto. Los gastos materiales de pago único, como pueden ser los ordenadores y periféricos de los empleados, no se contemplan dentro de este presupuesto.

\section{Justificación}

El teletrabajo se ha extendido por los departamentos de recursos humanos de muchas empresas. La incertidumbre a posibles nuevos eventos que nos obliguen a quedarnos en casa aumenta la necesidad de un sistema más flexible de trabajo. La empresa aprovecha esta circunstancia para invertir los recursos necesarios en flexibilizar la jornada laboral.

\textquote[\cite{herrera2021impacto}]{Cuando las demandas familiares interfieren en las responsabilidades laborales, se hace aún más interesante el uso de esta modalidad, disminuyendo el conflicto trabajo-familia.}
    
Varios estudios han demostrado que la flexibilidad laboral aumenta la productividad, lo que se traduce en una inversión a medio o largo plazo para la empresa.

Por otra parte, el FBI avisa en su informe anual (2021) que, desde la crisis causada por el SARS-CoV-2, los ataques informáticos han ido en un aumento constante. Demostrando que el bastionado de la infraestructura informática debe ser una prioridad.\cite{fbi2021ic3}